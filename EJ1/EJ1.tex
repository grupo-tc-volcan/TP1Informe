

\section{Ejercicio 1: Filtro notch pasivo}
\subsection{C\'alculo te\'orico}
Para el c\'alculo te\'orico se consider\'o al circuito como dos cuadripolos en paralelo, de los cuales se obtuvieron sus par\'ametros admitancia.
El primero de los cuadripolos es el presentado en la figura \todo[inline]{INSERT REFERENCE TO FIGURE Q1}.
\missingfigure{First quadrupole circuit}
El c\'alculo de los par\'ametros viene facilitado por la simpleza del circuito y el hecho de ser rec\'iproco, de forma que sus par\'ametros admitancia son:

\begin{align}
    \label{eqn: Admittance parameters first quadrupole.}
    y_{A11} = \left. \frac{I_1}{V_1} \right\rvert_{V_2=0} = \frac{1}{R_1 + \frac{R_2}{R_2 \cdot C_3 \cdot s + 1}} = \frac{R_2 \cdot C_3 \cdot s + 1}{R_1 \cdot R_2 \cdot C_3 \cdot s + \left(R_1+R_2\right)} \\
    y_{A12} = \left. \frac{I_1}{V_2} \right\rvert_{V_1=0} = \frac{-I_2 \cdot \frac{1}{R_1 \cdot C_3 \cdot s + 1}}{I_2 \cdot \left(R_2 + \frac{R_1}{R_1 \cdot C_3 \cdot s + 1}\right)} = -\frac{1}{R_1 \cdot R_2 \cdot C_3 \cdot s + \left(R_1+R_2\right)}\\
    y_{A21} = \left. \frac{I_2}{V_1} \right\rvert_{V_2=0} = -\frac{1}{R_1 \cdot R_2 \cdot C_3 \cdot s + \left(R_1+R_2\right)}\\
    y_{A22} = \left. \frac{I_2}{V_2} \right\rvert_{V_1=0} = \frac{R_1 \cdot C_3 \cdot s + 1}{R_1 \cdot R_2 \cdot C_3 \cdot s + \left(R_1+R_2\right)}
\end{align}

De forma an\'aloga se obtienen los par\'ametros para el segundo cuadripolo \todo[inline]{INSERT REFERENCE TO FIGURE Q2}, bas\'andose en los c\'alculos del primero, y tomando provecho de su similitud.
\missingfigure{Second quadrupole circuit}

\begin{align}
    \label{eqn: Admittance parameters second quadrupole.}
    y_{B11} = \frac{\frac{1}{R_3 \cdot C_2 \cdot s} + 1}{\frac{1}{R_3 \cdot C_1 \cdot C_2 \cdot s} + \frac{1}{C_1 \cdot s} + \frac{1}{C_2 \cdot s}} \\
    y_{B12} = -\frac{1}{\frac{1}{R_3 \cdot C_1 \cdot C_2 \cdot s} + \frac{1}{C_1 \cdot s} + \frac{1}{C_2 \cdot s}}\\
    y_{B21} = -\frac{1}{\frac{1}{R_3 \cdot C_1 \cdot C_2 \cdot s} + \frac{1}{C_1 \cdot s} + \frac{1}{C_2 \cdot s}}\\
    y_{B22} = \frac{\frac{1}{R_3 \cdot C_1 \cdot s} + 1}{\frac{1}{R_3 \cdot C_1 \cdot C_2 \cdot s} + \frac{1}{C_1 \cdot s} + \frac{1}{C_2 \cdot s}}
\end{align}