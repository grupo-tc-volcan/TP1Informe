% automatically generated document using lt2circuiTikz
%\documentclass[tikz,margin={2pt 2pt 2pt 2pt}]{standalone}
%\usepackage[compatibility,siunitx,  americanvoltages, americancurrents, europeanresistors, europeaninductors, americanports,%
%  straightlabels, fetbodydiode, straightvoltages]{circuitikz}
%\usepackage{tikz,amsmath, amssymb,bm,color,pgfkeys,siunitx,ifthen,ulem}
%\usepackage{pgfplots}
%\pgfplotsset{compat=1.14}
%\usetikzlibrary{shapes,arrows}
%\usepackage{agaramondc}					% Adobe Garamond, custom shape
%\renewcommand{\shapedefault}{rtl} % rtl: roman tabular lining

\makeatletter

%% bandstop filter (adapted from highpass)
\pgfcircdeclarebipole{}{\ctikzvalof{bipoles/highpass/width}}{*bandstop}{\ctikzvalof{bipoles/highpass/width}}{\ctikzvalof{bipoles/highpass/width}}{
	\pgf@circ@res@step = \ctikzvalof{bipoles/highpass/width}\pgf@circ@Rlen
	\divide \pgf@circ@res@step by 2
	
	\pgfpathmoveto{\pgfpoint{\pgf@circ@res@left}{\pgf@circ@res@zero}}
	\pgf@circ@res@other = \pgf@circ@res@left
	\advance\pgf@circ@res@other by \pgf@circ@res@step 
	
	\ifpgf@circuit@dashed
	\pgfsetdash{{0.1cm}{0.1cm}}{0cm} 
	\fi	
	
	% draw outer box
	\pgfsetlinewidth{\pgfkeysvalueof{/tikz/circuitikz/bipoles/thickness}\pgfstartlinewidth}
	\pgfpathrectanglecorners{\pgfpoint{\pgf@circ@res@left}{\pgf@circ@res@up}}{\pgfpoint{\pgf@circ@res@right}{\pgf@circ@res@down}}
	\pgfusepath{draw}
	
	\ifpgf@circuit@inputarrow
	{
		\advance \pgf@circ@res@left by -.5\pgfkeysvalueof{/tikz/circuitikz/bipoles/thickness}\pgfstartlinewidth
		\pgftransformshift{\pgfpoint{\pgf@circ@res@left}{0pt}}
		\pgfnode{inputarrow}{tip}{}{pgf@inputarrow}{\pgfusepath{fill}}
	}
	\fi
	
	% rotate inner symbol
	\def\pgfcircmathresult{\expandafter\pgf@circ@stripdecimals\pgf@circ@direction\pgf@nil}
	\ifnum \pgfcircmathresult > 45 \ifnum \pgfcircmathresult < 135
	\pgftransformrotate{270}
	\fi\fi
	\ifnum \pgfcircmathresult > 134 \ifnum \pgfcircmathresult < 225  % 134 degree, because >= 135 is not possible
	\pgftransformrotate{180}
	\fi\fi
	\ifnum \pgfcircmathresult > 224 \ifnum \pgfcircmathresult < 315
	\pgftransformrotate{90}
	\fi\fi
	
	% draw inner symbol
	\pgfsetdash{}{0pt}	% always draw solid line for inner symbol
	\pgfsetarrows{-} %never draw arrows
	\pgfsetlinewidth{\pgfstartlinewidth}
	\pgfpathmoveto{\pgfpoint{-0.5\pgf@circ@res@step}{0.5\pgf@circ@res@step}}
	\pgfpathsine{\pgfpoint{.25\pgf@circ@res@step}{.25\pgf@circ@res@step}}
	\pgfpathcosine{\pgfpoint{.25\pgf@circ@res@step}{-.25\pgf@circ@res@step}}
	\pgfpathsine{\pgfpoint{.25\pgf@circ@res@step}{-.25\pgf@circ@res@step}}
	\pgfpathcosine{\pgfpoint{.25\pgf@circ@res@step}{.25\pgf@circ@res@step}}
	\pgfusepath{draw}
	
	\pgfpathmoveto{\pgfpoint{-0.5\pgf@circ@res@step}{0}}
	\pgfpathsine{\pgfpoint{.25\pgf@circ@res@step}{.25\pgf@circ@res@step}}
	\pgfpathcosine{\pgfpoint{.25\pgf@circ@res@step}{-.25\pgf@circ@res@step}}
	\pgfpathsine{\pgfpoint{.25\pgf@circ@res@step}{-.25\pgf@circ@res@step}}
	\pgfpathcosine{\pgfpoint{.25\pgf@circ@res@step}{.25\pgf@circ@res@step}}
	\pgfusepath{draw}
	\pgfpathmoveto{\pgfpoint{-0.15\pgf@circ@res@step}{-0.15\pgf@circ@res@step}}
	\pgfpathlineto{\pgfpoint{0.15\pgf@circ@res@step}{0.15\pgf@circ@res@step}}
	\pgfusepath{draw}
	
	\pgfpathmoveto{\pgfpoint{-0.5\pgf@circ@res@step}{-0.5\pgf@circ@res@step}}
	\pgfpathsine{\pgfpoint{.25\pgf@circ@res@step}{.25\pgf@circ@res@step}}
	\pgfpathcosine{\pgfpoint{.25\pgf@circ@res@step}{-.25\pgf@circ@res@step}}
	\pgfpathsine{\pgfpoint{.25\pgf@circ@res@step}{-.25\pgf@circ@res@step}}
	\pgfpathcosine{\pgfpoint{.25\pgf@circ@res@step}{.25\pgf@circ@res@step}}
	\pgfusepath{draw}
	%	\pgfpathmoveto{\pgfpoint{-0.15\pgf@circ@res@step}{-0.65\pgf@circ@res@step}}
	%	\pgfpathlineto{\pgfpoint{0.15\pgf@circ@res@step}{-0.35\pgf@circ@res@step}}
	%	\pgfusepath{draw}
}

\tikzset{
	*bandstop/.style={\circuitikzbasekey, /tikz/to path=\pgf@circ@*bandstop@path},
}
\def\pgf@circ@*bandstop@path#1{\pgf@circ@bipole@path{*bandstop}{#1}}




\makeatother

\usetikzlibrary{backgrounds,calc,positioning}

\usetikzlibrary{circuits.ee.IEC}
\usetikzlibrary{arrows}


% sym32a style

\ctikzset{tripoles/mos style/arrows}
\ctikzset{
	/tikz/circuitikz/quadpoles/coupler/width=1,%1.3
	/tikz/circuitikz/quadpoles/coupler/height=0.952,%1.3
	/tikz/circuitikz/quadpoles/coupler2/width=1,%1.3
	/tikz/circuitikz/quadpoles/coupler2/height=0.952,%1.3
	/tikz/circuitikz/quadpoles/transformer/width=1.425,%1.5
	/tikz/circuitikz/quadpoles/transformer/height=1.425,%1.5
	/tikz/circuitikz/quadpoles/transformer core/width=1.425,%1.5
	/tikz/circuitikz/quadpoles/transformer core/height=1.425,%1.5
	/tikz/circuitikz/quadpoles/gyrator/width=1.425,%1.5
	/tikz/circuitikz/quadpoles/gyrator/height=1.425,%1.5
	/tikz/circuitikz/monopoles/tlinestub/width=0.1875,%0.25 no effect!
	/tikz/circuitikz/tripoles/american and port/height=0.95,%.8
	/tikz/circuitikz/tripoles/american nand port/height=0.95,%.8
	/tikz/circuitikz/tripoles/american or port/height=0.95,%.8
	/tikz/circuitikz/tripoles/american nor port/height=0.95,%.8
	/tikz/circuitikz/tripoles/american xor port/height=0.95,%.8
	/tikz/circuitikz/tripoles/american xnor port/height=0.95,%.8
	/tikz/circuitikz/bipoles/tline/height=0.4,%0.3
%	/tikz/circuitikz/bipoles/tline/width=1.2,%0.8
	/tikz/circuitikz/bipoles/diode/height=0.375,%
	/tikz/circuitikz/bipoles/diode/width=0.375,%
	/tikz/circuitikz/bipoles/varcap/height=0.375,%
	/tikz/circuitikz/bipoles/varcap/width=0.375,%
	/tikz/circuitikz/tripoles/triac/height=1.05,%
	/tikz/circuitikz/tripoles/triac/width=0.952,%
	/tikz/circuitikz/tripoles/thyristor/height=1.05,%
	/tikz/circuitikz/tripoles/thyristor/width=0.952,%
	/tikz/circuitikz/tripoles/op amp/height=0.952,%
	/tikz/circuitikz/tripoles/op amp/width=1.2,%
	/tikz/circuitikz/tripoles/op amp/font=\footnotesize,
	/tikz/circuitikz/tripoles/gm amp/height=0.952,% 1.7
	/tikz/circuitikz/tripoles/gm amp/width=1.2,% 1.4
	%	/tikz/circuitikz/tripoles/gm amp/font=\footnotesize,
	/tikz/circuitikz/tripoles/plain amp/height=0.952,% 1.7
	/tikz/circuitikz/tripoles/plain amp/width=1.2,% 1.4
	/tikz/circuitikz/bipoles/resistor/voltage/straight label distance/.initial=.8,
	/tikz/circuitikz/bipoles/generic/voltage/straight label distance/.initial=.8,
	/tikz/circuitikz/bipoles/inductor/voltage/straight label distance/.initial=.8,
	/tikz/circuitikz/bipoles/fullgeneric/voltage/straight label distance/.initial=.8,
	/tikz/circuitikz/bipoles/capacitor/voltage/straight label distance/.initial=1.0,
	/tikz/circuitikz/bipoles/thickness=1.6,
}
\ctikzset{v/.append style={/tikz/european voltages}}

\definecolor{netlabelcolor}{rgb}{0, 0, 0.25}
\definecolor{lttotitextcolor}{rgb}{0, 0.4, 0.25}
\definecolor{lttotidrawcolor}{rgb}{0.6, 0.6, 0.6}
\definecolor{netcolor}{rgb}{0, 0, 0.5}

\pgfkeys{/lt2ti/netlabel/font/.initial= \small}
\pgfkeys{/lt2ti/text/font/.initial= \small}

\pgfkeys{/lt2ti/Net/.style= {netcolor}}
\tikzstyle{dashdotdotted}=[dash pattern=on 3pt off 2pt on \the\pgflinewidth off 2pt on \the\pgflinewidth off 2pt]

\pgfkeys{/lt2ti/VArrow/.style= {->,>=latex}}
\pgfkeys{/lt2ti/SArrow/.style= {->,>=angle 90}}

\begin{figure}[h]
	\centering
	\begin{tikzpicture}[circuit ee IEC, scale=0.5,line width=.5pt]% default: 0.4
		%\tikzstyle{every node}=[font=\small];%
		%\node [draw] at (0.0,0.0) {\pgfkeysvalueof{/tikz/circuitikz/tripoles/op amp/font}};
		\draw [/lt2ti/Net](3.0,-2.5)to[*short,-, color=netcolor] (-0.5,-2.5);% wire w3
		\draw [/lt2ti/Net](7.0,-2.5)to[*short,*-, color=netcolor] (5.0,-2.5);% wire w4
		\draw [/lt2ti/Net](9.0,-2.5)to[*short,-*, color=netcolor] (7.0,-2.5);% wire w5
		\draw [/lt2ti/Net](14.0,-2.5)to[*short,-, color=netcolor] (11.0,-2.5);% wire w6
		\draw [/lt2ti/Net](7.0,-5.0)to[*short,-*, color=netcolor] (7.0,-2.5);% wire w7
		\draw [/lt2ti/Net](7.0,-9.0)to[*short,*-, color=netcolor] (7.0,-7.5);% wire w8
		\draw [/lt2ti/Net](7.0,-9.0)to[*short,*-, color=netcolor] (7.0,-9.0);% wire w9_w11 start
		\draw [/lt2ti/Net](-0.5,-10.0)to[*short,-, color=netcolor] (-0.5,-10.0);% wire w9_w11 end
		\draw [/lt2ti/Net](7.0,-9.0) --  (-0.5,-9.0) -- (-0.5,-10.0); % wire w9_w11 polyline 
		\draw [/lt2ti/Net](14.0,-10.0)to[*short,-, color=netcolor] (14.0,-10.0);% wire w10_w12 start
		\draw [/lt2ti/Net](7.0,-9.0)to[*short,-*, color=netcolor] (7.0,-9.0);% wire w10_w12 end
		\draw [/lt2ti/Net](14.0,-10.0) --  (14.0,-9.0) -- (7.0,-9.0); % wire w10_w12 polyline 
		\draw (-0.5, -10.0) node[rground, xscale=1, yscale=1, rotate=0, ] (undefined) {};%  (undefined)++(0.0,0.0) node {undefined }; % component "circuiTikz\gnd" "undefined" 
		\draw (14.0, -10.0) node[rground, xscale=1, yscale=1, rotate=0, ] (undefined) {};%  (undefined)++(0.0,0.0) node {undefined }; % component "circuiTikz\gnd" "undefined" 
		\draw (5.0, -2.5) to[*capacitor, l^=C1, a_=, -, ] (3.0,-2.5){}; % component "cap" "C1" 
		%\node [] at (5.0,-2.0) {x}; % component "cap" "C1" 
		\draw (11.0, -2.5) to[*capacitor, l^=C2, a_=, -, ] (9.0,-2.5){}; % component "cap" "C2" 
		%\node [] at (11.0,-2.0) {x}; % component "cap" "C2" 
		\draw (7.0, -5.0) to[*resistor, l^=R3, a_=, -, ] (7.0,-7.5){}; %\node [] at (6.5,-4.5) {x}; % component "res" "R3" 
		\node (Vi) [] at (-0.5,-2.5) {};% label mark % label "" "Vi" lbl15 
		\node (Vitxt) [ netlabelcolor, above= -0.24cm of Vi] {{\pgfkeysvalueof{/lt2ti/netlabel/font}Vi}}; % label "" "Vi" lbl15 
		\node (Vo) [] at (14.0,-2.5) {};% label mark % label "" "Vo" lbl16 
		\node (Votxt) [ netlabelcolor, above= -0.24cm of Vo] {{\pgfkeysvalueof{/lt2ti/netlabel/font}Vo}}; % label "" "Vo" lbl16 

	\end{tikzpicture}

	\caption{Cuadripolo B}
	\label{fig: Second Quadrupole}
\end{figure}

